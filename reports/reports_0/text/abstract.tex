% \onecolumn
\section{Abstract}
\subsection{Background}%
Outcomes for hospitalized patients, including metrics such as length of stay
and early readmission, are difficult to predict.\@
Hospital systems and governments have focused on improving these metrics,
with uncertain benefit.\@
Whether interpretable machine learning algorithms can simultaneously
achieve useful predictive power and identify actionable 
patient-specific and institutional variables remains unknown.\@
\subsection{Methods}%
We collected clinical, demographic, and institutional data for all hospitalizations
from January 2011 to May 2018.\@
Summary statistics were used to describe the cohort.\@
Machine learning algorithms were trained to predict 
readmission, length of stay, in-hospital death within 48--72 hours, and demographic data.\@
Model performance was evaluated using a variety of metrics, including 
area under the receiver operator characteristic curve (AUC) and Brier score loss (BSL).\@
% precision-recall curves, average precision root mean square error,
% median absolute error, mean absolute error, and R\textsuperscript{2} scores.\@
Global and individual prediction explanations 
were generated using Shapley values and visualizations.\@
\subsection{Results}%
Nearly 1.5 million hospitalizations for over 700,000 unique patients were used
for model development, validation, and testing. 
Readmission rates were higher for patients who were 
older, black, divorced/separated, on Medicare, 
and with high overall disease burden.\@
Thirty-day readmission was predicted with an AUC of 0.76 and BSL of 0.1.\@
Length of stay greater than 5 days was predicted with an AUC of 0.85 and BSL of 0.14.\@
% Length of stay was predicted with an RMSE of 3.97 days if treated continuously,
% and an AUC of 0.85 and Brier score loss of 0.14 
% if binarized into length of stay greater or less than 5 days.\@
Death within 48--72 hours was predicted with an AUC of 0.81.\@ 
% but was poorly calibrated due to class imbalance.\@
Performance metrics for other targets were AUC 0.87 and BSL 0.14 (gender), AUC 0.91 and BSL 0.10 (race), 
AUC 0.92 and BSL 0.10 (financial class), and AUC 0.94 and BSL 0.09 (age greater than 65 years).\@
Explanatory diagrams for individuals, the cohort, and variable interactions were produced for each predictive target.\@ 
% Gender, race, and financial class were predicted with AUCs and Brier score losses of
% 0.87 and 0.14, 0.91 and 0.10, and 0.92 and 0.10, respectively.\@
% Age greater than 65 years was predicted with an AUC of 0.94 and Brier score loss of 0.09.\@
% with an RMSE of 10 years if treated as continuous, 
% and with an AUC of 0.94 and Brier score loss of 0.09 
% if binarized into older or younger than 65 years.\@
\subsection{Conclusions}%
Interpretable machine learning algorithms can achieve state-of-the-art 
predictive power for hospital readmission and extended length of stay,
while simultaneously providing locally and globally interpretable predictions
that account for clinical, demographic, and institutional variables.\@
% They also identify key demographic features that may 
% reveal sources of healthcare inequality.\@
\newpage%