\section{Discussion}
%summary of why
\subsubsubsection{{Justification}}
Our investigation of machine learning methods for 
predicting and explaining inpatient outcomes
was initiated as a result of 
1) increased focus on the inpatient experience in the United States and other countries, 
2) availability of complex data in the EHR,
and 3) the development of explainable predictive models.\@
Additionally, recent concerns over the impact of metrics such as 
readmission rates yield an opportunity to develop models that may be used to 
not only predict, but also understand the components of risk and their interactions.\supercite{Wadhera2018}
We therefore sought to predict and understand current and future readmissions, predict and understand length of stay,
and, further, examine health care disparities.\@ 

The desired decision-making aid for patients, providers, and hospital systems
is a robust and reliable prediction for an outcome of interest,
paired with an explanation of the components of the predicted probability.\@
We produce per-encounter probability estimates, which can be compared against the calibration curve
to determine the reliability of the estimate. The most important components
of this prediction can then be examined, which would ideally lead to 
action items that can be modified to attempt risk optimization,\supercite{Donze2013, Leppin2014, Burke2017, Auerbach2016}
or at least to a deeper understanding of the current situation.\supercite{Saunders2015}
We also generate cohort-level diagrams that explain the 
contributions of each variable to the model output
as well as key variable interactions.\@

% sweeping overview
\subsubsubsection{{Summary}}
Our models achieved comparable performance to the existing state of the art 
in the prediction of readmission and length of stay.\supercite{Rajkomar2018, Aubert2017, Artetxe2018}\@
In addition to reporting AUC, which conveys performance across classification cutoffs,
we reported that our models are well-calibrated when using raw probabilities, 
which may be more useful than binary classifications in many settings.\supercite{Steyerberg2010}
Additionally, we generated per-encounter and per-cohort explanations for these models.
Using these explanations in combination with the probability reliability estimate, 
we are able to audit the model on local and global levels, 
as well as generate further hypotheses for exploration through examination of 
variable interactions.\@
We also used these techniques to explore healthcare disparities 
by making demographic variables into predictive targets, 
followed by global, local, and variable interaction explanations.\@
\subsubsubsection{{Innovations}}
% Explainability

% Dirty data is good data
Because of the focus on interpretability, the study was designed to cast a broad net
with regards to variable and patient inclusion criteria.\@
Rather than including only CMS-defined readmissions, 
we chose to include patients of all ages and diagnos:es who survived the index hospitalization,
including those on observation status.\supercite{obsincreasing2019, obs2016nejm}\@
This allowed us to examine the impacts of these variables, as well as develop a broadly applicable
model for the institution as a whole, which includes many specialties, hospitals, and a range of socioeconomic environs.\@
We also observed variable interactions that would not have been otherwise apparent, 
such as the interplay between age and heart rate for 30-day readmissions (\ref{fig:30dinthrage}),
which would not have been found if the age was restricted to over 65 as in CMS criteria.
In another example, we found that leaving against medical advice (a criterion for disinclusion in CMS-defined readmission)
has comparatively little impact on readmission risk,
but that a discharge to hospice has a downward impact on readmission risk that 
varies with a patient's number of past hospitalizations (\ref{fig:30dintdispoadmits}).\@

% why GBMs
EHR data is typically heterogenous and incomplete.
We chose to use GBMs because they perform well with this kind of data.\@
% no scaling or embedding
For example, because they do not require feature scaling,
GBM explanations of predictions use metrics with the same units as the input 
(e.g.\ red blood cell count, heart rate, BMI).
They can also account for categorical variables with many levels (e.g.\ diagnoses), 
and do not combine variables in an ad hoc and inseparable fashion (as is the case with feature embedding).\@
The explanations are therefore easily readable.\@
% missingness and imputation
Additionally, prior work with EHR data has shown that the presence or absence of a laboratory result can itself be a useful predictor,
even if the actual value of the laboratory test is ignored.\supercite{Agniel2018}
GBMs do not require imputation of missing values 
nor removal of examples with missing data, and so allowed us to capture the impact of missingness.\@
% parsimonious models
Lastly, GBMs, particularly in combination with an explanatory method such as SHAP,
allow for robust selection of the most impactful variables.\@
Feature selection using SHAP considers all variables in every possible combination, and so produces reliable, consistent explanations
and measures of the impact of individual variables.\@
We found that a model for readmission that included only the top 10 SHAP variables
was equally performant to a model with the full feature set.\@
The smaller model may reduce the richness of individual
explanations, but a parsimonious approach may be more applicable in settings without 
broad and deep EHR data availability, or without the ability to embed a complex model.\@
\subsubsubsection{{Caveats}}
% caveats
The study was not designed to recapitulate or examine CMS criteria for readmission.\@
Without further restriction of our study population to meet CMS criteria,
it would not be appropriate to generalize our models or explanations for CMS applications.\@
The general approach, however, can easily be adapted to a more curated cohort.\@
Additionally, our variables were selected based on existing literature and available in-house data.
This is both a strength and a limitation, 
as selection of specific variables does not require the EHR to adopt a standard 
such as the Fast Healthcare Interoperability Resources (FHIR) specification,
but also cannot account for variables it was not explicitly given.
Artetxe et al.\ achieved comparable results to ours using a FHIR-based 
data collection method and a deep learning model trained on nearly 
47 billion data points spread over 200,000 patients.\supercite{Artetxe2018}
Aubert et al.\ also achieved comparable results using a model with only 6 variables.\supercite{Aubert2017}
Our study was designed to maximize interpretability, and it may be the case that, using our larger cohort,
deep learning or less complex approaches would achieve similar or superior predictive power.\supercite{unreasonable2017}
\subsubsubsection{{Limitations}}
% limitations
The study has several limitations.
% beginning/end variables
First, we selected only variables available at the beginning and end of the hospitalization.\@
A model that accounts for and explains daily changes in risk may be a useful tool for dynamic discharge planning, etc.,
but would require a different approach and greatly increase required processing time.\@ 
% primary diagnosis != most important diagnosis?
Second, we included only the primary diagnosis for each patient. 
Selection of the primary diagnosis is difficult to disentangle from insurance concerns.
However, despite this we found that primary diagnosis was among the most important variables in 
several prediction models, and seems to consistently yield useful information. 
Additionally, as primary diagnosis is selected from among all patient diagnoses
at the end of the hospital encounter, there may be cases where it does not 
reflect the admission diagnosis (which was not available as a discrete variable in our database).\@ 
This may weaken the applicability of the length of stay prediction.\@
% readmissions to our hospital
Third, because we only used data available in our EHR, 
we could only assess for readmissions to our hospital system.\@
We therefore did not capture the total readmission rate,
nor could we account for admissions to our system that were readmissions
from another system.\@ 
We also only included EHR data available for a given patient within the timeframe of the study,
rendering variables such as ``number of past admissions'' valid only within the years specified.\@
Lastly, though we are able to extract important variables and examine their impact for individuals and the cohort,
certain variables are, in themselves, difficult to interpret.\@
For example, the repeatedly important variable ``ACS BlockGroup'' may have been suggesting, at least in some cases,
that the patient simply lives nearby and is likely to come back, or will stay instead of transferring closer to home, and so forth.\@
Deeper exploration of census data on the outcomes of interest, 
in addition to hospital- and patient-specific factors, may help elucidate the impacts of 
geographic, socioeconomic, and demographic risk factors.\supercite{krumholz2017}
\subsubsubsection{{Further Directions}}
% further directions
This study raises several questions and suggests further avenues for research.\@
As this was a retrospective study, we are beginning validation of this model
in a prospective study in our hospital system.\@
Additionally, because this was a single-system study, it requires validation in other systems.\@
Further, a selection of subpopulations may allow richer explanations and, 
possibly, increased predictive power.\supercite{Yu2015, Smith2018, Weinreich2016, Garrison2017, Golas2018}
\subsubsubsection{{Conclusion}}
%conclusion
In conclusion, we generated locally and globally explainable prediction models 
that reliably predict the probability of readmission and length of stay,
as well as sociodemographic factors including gender, race, financial class, and age.\@
We propose the use of this approach as an auditable decision aid that 
also contributes to hypothesis creation.\@

%%%%%%%%%%%%%%%%%%%%%%%%%%%%%%%%%%%%%


% Resources:
% \supercite{Aubert2017}, % Simplification of the HOSPITAL score forpredicting30-dayreadmissions.
% \supercite{Ephrem2013}, % RDW FTW.
% \supercite{Auerbach2016}, % preventability and Causes of Readmissions in a National Cohort of General Medicine Patients.
% \supercite{Burke2017}, % The HOSPITAL Score Predicts Potentially Preventable 30-Day Readmissions in Conditions Targeted by the Hospital Readmissions Reduction Program
% \supercite{Donze2013}, %Potentially Avoidable 30-Day Hospital Readmissions in Medical Patients: derivation and Validation of a Prediction Model.
% \supercite{Golas2018}, %A machine learning model to predict the risk of 30-day readmissions in patients with heart failure: a retrospective analysis of electronic medical records data
% \supercite{Leppin2014}, %  Preventing 30-Day Hospital Readmissions: A Systematic Review and Meta-analysis of RandomizedTrials.
% \supercite{Saunders2015}, % Examination of unplanned 30-day readmissions to a comprehensive cancer hospital

% Unused so far:
% \supercite{Allaudeen2011}, % Inability of Providers to Predict Unplanned Readmissions.
% \supercite{Collins2015}, % Transparent Reporting of a multivariable prediction model for Individual Prognosis Or Diagnosis (TRIPOD): The TRIPOD Statement


%%%%%%%%%%%%%%%%%%%%%%%%%%%%%%%%%%%%%