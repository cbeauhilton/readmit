\twocolumn
\section{Introduction}
\sloppy
% \paragraph{}
\lettrine[lines=3,lraise=0.05,nindent=0pt]{\textcolor{NEJMGreyText}{P}}{atients} and providers
face a great amount of uncertainty before, during, and after hospital encounters.\@
Predictive modeling holds promise for identifying patients at highest risk for adverse events, %
such as extended length of stay, 30-day readmission, and death within the hospital encounter.\@
Despite the success of predictive models in achieving discriminatory power in these and other areas, %
simplistic models cannot account for complicated intersections of medical, %
institutional, and demographic factors. Conversely, %
complex models that account for these interactions are difficult or impossible to interpret or audit.\@
They may therefore be inactionable or harmful if put into use.\@
Recent studies suggest that a focus on metrics such as readmission without addressing
underlying causes may lead to patient harm.\supercite{Wadhera2018} \@ 
New techniques in interpretable machine learning allow for %
complex models that are globally and locally interpretable, and may %
therefore be useful for prediction as well as exploration of the underlying phenomena.\@

We hypothesized that interpretable predictive models would achieve comparable or superior performance to 
existing models, and enable an understanding of factors leading to adverse outcomes for hospitalized patients.\@
We also hypothesized that these models would reveal disparities in outcomes due to demographic factors,
and show interactions between key variables. Here, we report models with high predictive power for readmission and
extended length of stay, along with local and global interpretations, and discuss the use of 
machine learning as a tool to aid understanding and find actionable insights.