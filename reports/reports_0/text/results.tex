\section{Results}
\subsubsection{Study Cohort}
\sloppy
\input{"pygen/results_paragraphs_latex.txt"}
% \hyperref[table:table1]{Table 1}
% \hyperref[table:table2]{Table 2}

\subsubsection{Prediction of Inpatient Outcomes}
Summary performance metrics are shown in \hyperref[table:table2]{Table 2}.
\subsubsubsection{Thirty-day unplanned readmission}
Thirty-day readmissions were predicted with an AUC of 0.76 (Fig.\ \ref{fig:roc30d}).\@
The Brier score loss was 0.1, calibration curve shown in Fig.\ \ref{fig:cal30d}.\@
The most impactful features included primary diagnosis, 
time between the start of the current admission and the previous admission,
discharge disposition location, 
number of past admissions, length of stay, number of reported comorbidities,  
total emergency department visits in the last six months,
discharge disposition, admission source, and BMI on admission and discharge, 
as well as others (Figs.\ \ref{fig:shapsumbar30d} and~\ref{fig:shapsum30d}).\@
Including more than the top ten variables in the model did not improve predictive power for the cohort overall, 
but does allow for more specific rationale for prediction for certain patients, as well as 
examination of feature interactions for further exploration.\@
Sample individualized predictions with their explanations are shown in 
Figs.\ \ref{fig:shapforce30d1},~\ref{fig:shapforce30d2}, and~\ref{fig:shapforce30d3}.\@

In order to examine possible changes in causes of readmission risk as a function of time from discharge, 
we predicted readmission risk for several readmission thresholds and calculated Shapley values for each.\@
Shapley values for 3-day and 7-day readmission are shown in Figs.\ \ref{fig:shaprdt3d} and~\ref{fig:shaprdt7d}, respectively.\@
For example, 7-day readmission risk prediction achieved AUC of 0.70 %
with a Brier score loss of 0.04 (\hyperref[table:table2]{Table 2}).\@
The most impactful feature remained primary diagnosis, %
but insurance provider, census block group, and other demographic features played more important roles, %
along with other variables shown in Fig.\ \ref{fig:suppshapfig}.\@

\subsubsubsection{Length of stay}
Length of stay was predicted in terms of number of days and was binarized at various thresholds.\@
Length of stay in days was predicted within 3.97 days measured by root mean squared error.\@
Length of stay over 5 days was predicted with an AUC of 0.85 (Fig.\ \ref{fig:roclos5d}), 
with a PPV for length of stay less than 5 days of 0.82, 
and a Brier score loss of 0.14 (calibration curve shown in Fig.\ \ref{fig:callos5d}).\@ 
The most impactful features for the regression and classification problems were similar, 
and included the type of admission, primary diagnosis code, patient age, admission source, 
length of stay of the most recent prior admission, 
medications administered in the hospital in the first 24 hours, insurance, 
and early admission to the intensive care unit, among others shown in Figs.\ \ref{fig:shapsumbarlos5d} and~\ref{fig:shapsumlos5d}.\@
Impactful features for length of stay at thresholds of 3 and 7 days are shown in %
Figs.\ \ref{fig:shaplos3d} and~\ref{fig:shaplos7d}, respectively.\@ 

\subsubsubsection{Death within 48\textendash72 hours}
Prediction of death within 48--72 hours of admission was predicted with an AUC of 0.81 (\hyperref[table:table2]{Table 2}). 
However, due to extreme class imbalance (e.g.\ in the testing set there were 260,518 non-deaths and 390 deaths), 
this was achieved by simply predicting non-death in every case. Strategies to address class imbalance
were unsuccessful. More work is needed in this area before reliable model interpretations are produced. 

\subsubsection{Prediction of Patient Characteristics}
\subsubsubsection{{Gender}}
Gender prediction achieved an AUC of 0.87 (Fig.\ \ref{fig:aucgender}),\@
and a Brier score loss of 0.14 (calibration curve shown in Fig.\ \ref{fig:calgender}).\@
The most impactful features for gender prediction are shown in Fig.\ \ref{fig:shapsumgender}.\@

\subsubsubsection{{Race}}
Race prediction (white vs.\ non-white) achieved an AUC of 0.91 (Fig.\ \ref{fig:aucrace}), 
and a Brier score loss of 0.1 (calibration curve shown in Fig.\ \ref{fig:calrace}).\@
The most impactful features for racial prediction are shown in Fig.\ \ref{fig:shapsumrace}.\@

\subsubsubsection{{Financial Class}}
Financial class prediction (Medicare/Medicaid vs.\ private insurance) achieved an AUC of 0.92 (Fig.\ \ref{fig:aucinsurance}), 
and a Brier score loss of 0.1 (calibration curve shown in Fig.\ \ref{fig:calinsurance}).\@
The most impactful features for insurance prediction are shown in Fig.\ \ref{fig:shapsuminsurance}.\@

\subsubsubsection{{Age}}
When predicted as a continuous variable, age was only able to be predicted within about 10 years (\hyperref[table:table2]{Table 2}).\@
If age was binarized into patients older or younger than 65 years, 
prediction achieved an AUC of 0.94 (Fig.\ \ref{fig:aucage}) and Brier score loss of 0.09 (Fig.\ \ref{fig:calage}).\@
The most impactful features for age prediction are shown in Fig.\ \ref{fig:shapsumage}.\@

\subsubsection{{Variable Interactions}}
SHAP analysis also allows examination of interactions between variables.\@
Key variable interactions are shown in Figs.\ \ref{fig:30dint},~\ref{fig:los5dint},%
~\ref{fig:ageint},~\ref{fig:raceint},~\ref{fig:genderint}, and~\ref{fig:insuranceint}.\@ 
For example, high and low values of heart rate were shown to affect probability of readmission
differently for patients at different ages. A lower heart rate, between 50 and 100 beats per minute,
reduced the likelihood of readmission for older patients, while a high heart rate increased it.\@
For children, a high heart rate had little impact on the model output (Fig.\ \ref{fig:30dinthrage}).\@
For prediction of length of stay over 5 days, for middle-aged patients the number of medications
administered in the first 24 hours had little impact on model output, whereas very low or very high
numbers of medications greatly decreased or increased the prediction of length of stay, respectively,
for older and younger patients (Fig.\ \ref{fig:los5dintagemeds}).\@
