\twocolumn
\section{Introduction}
\sloppy
% \paragraph{}
\lettrine[lines=3,lraise=0.05,nindent=0pt]{\textcolor{NEJMGreyText}{P}}{atients} and providers
face a great amount of uncertainty before, during, and after hospital encounters.\@
Predictive modeling holds promise for identifying patients at highest risk for adverse events, 
such as extended length of stay (LOS), 30-day readmission, and death within the hospital encounter.
Despite the success of predictive models in achieving discriminatory power in these and other areas,
simplistic models cannot account for complicated intersections of medical, institutional, and demographic factors. 
Conversely, complex models that account for these interactions are difficult or impossible to interpret or audit, 
and therefore may be inactionable or harmful if put into use, 
and can also be difficult for healthcare providers understand or accept. \supercite{auerbach2018balancing,cabitza2017unintended,sniderman2015role}
Recent studies suggest that a focus on metrics such as 30-day readmission without addressing underlying causes 
may lead to increased patient mortality and increased cost without improving patient outcomes. \supercite{Wadhera2018}

Significant recent advances in artificial intelligence (AI), machine learning (ML), and deep learning (DL) 
have yielded compelling innovations from self-driving cars \supercite{bojarski2016end} to optimizing online search,\supercite{agirre2009personalizing}
product recommendations,\supercite{bobadilla2013recommender}
and beating the world champions in complex games such as chess and Go.\supercite{silver2018general}
These advances have also started to impact healthcare: 
detecting diabetic retinopathy in ophthalmology images,\supercite{gulshan2016development}
detecting cancers in biopsy slides,\supercite{coudray2018classification} 
and identifying malignant versus benign skin lesions
with accuracy comparable to or exceeding trained physicians.\supercite{esteva2017dermatologist} 
As electronic healthcare record data increase in size and complexity, 
AI and ML may open opportunities to provide predictive modeling that can improve 
patient safety and outcomes while decreasing healthcare cost. 
A major hurdle to implement ML algorithms in healthcare is the “black box phenomenon,” 
or lack of explainability of these models to physicians and healthcare providers. 
However, recent advances have provided algorithms that extract important variables and explain model decisions. 
Such an approach can ensure that variables included in the final model are clinically relevant 
and can be recognized and understood by patients and healthcare providers. 

In this study, we hypothesized that interpretable predictive models using ML would achieve comparable 
or superior performance to existing models, and enable an understanding of factors leading to adverse 
outcomes for hospitalized patients. 
We also hypothesized that these models would reveal disparities in outcomes due to demographic factors, 
and show interactions between key variables. 
Here, we report ML models with high predictive power for readmission and extended LOS, 
along with local and global interpretations, and discuss the use of ML as a tool to aid understanding and find actionable insights.
