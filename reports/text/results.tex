\section{Results}
\subsubsection{Study Cohort}
% \sloppy
% \input{"pygen/results_paragraphs_latex.txt"}
% \hyperref[table:table1]{Table 1}
% \hyperref[table:table2]{Table 2}
In the study period there were 1,485,880 hospitalizations 
for 708,089 unique patients, 439,696 (62\%) of whom had only one hospitalization recorded. 
The median number of hospitalizations per patient was 1 (QI [1.0, 2.0]). 
There were 211,022 thirty-day readmissions for an overall readmission rate of 14\%. 
Among patients aged >65 years, the 30-day readmission rate was 16\%. 
The median LOS, including patients in observation status and labor and delivery patients, 
was 2.94 days (QI [1.67, 5.34 ]). 
The demographic and clinical characteristics of the patient cohort are summarized in \hyperref[table:table1]{Table 1}. 
Higher rates of 30-day readmissions were observed in patients who were 
older (median age 62 vs. 59 years), African American (rate of 17\% vs. 13\% in whites), 
divorced/separated or widowed (17\% vs. 13\% in married/partnered or single patients), 
on Medicare insurance (rate of 17\% vs. 10\% for private insurance), 
and had one or multiple chronic conditions such as 
cancer, renal disease, congestive heart failure, and chronic obstructive pulmonary disease, etc.\ % \hyperref[table:table1]{(Table 1)}

\subsubsection{Prediction of Inpatient Outcomes}
\subsubsubsection{Thirty-day unplanned readmission}
Thirty-day readmissions were predicted with an AUC of 0.76 (Fig.\ \ref{fig:roc30d}).\@
The BSL was 0.1, calibration curve shown in Fig.\ \ref{fig:cal30d}.\@
The most impactful features included (ranked from most to least important): 
primary diagnosis, time between the start of the current admission and the previous discharge, 
discharge disposition location, number of past admissions, LOS, number of reported comorbidities, 
total emergency department visits in the last six months, discharge disposition,
admission source, and Body Mass Index (BMI) on admission and discharge, 
as well as others (Figs.\ \ref{fig:shapsumbar30d},~\ref{fig:shapsum30d}).\@
Including more than the top ten variables in the model did not improve predictive power for the cohort overall, 
but does allow for more specific rationale for prediction for certain patients, as well as 
examination of feature interactions for further exploration.\@
Sample individualized predictions with their explanations are shown in 
Figs.\ \ref{fig:shapforce30d1},~\ref{fig:shapforce30d2}, and~\ref{fig:shapforce30d3},
and in the Supplementary Appendix.\@

In order to examine possible changes in causes of readmission risk as a function of time from discharge, 
we predicted readmission risk for several readmission thresholds and calculated Shapley values for each.\@
Shapley values for 3-day and 7-day readmission are shown in Figs.\ 
\ref{fig:shaprdt3d} and~\ref{fig:shaprdt7d}, respectively.\@
For example, 7-day readmission risk prediction achieved AUC of 0.70 %
with a BSL of 0.05 (\hyperref[table:table2]{Table 2}).\@
The most impactful feature remained primary diagnosis, %
but insurance provider, demographic census block group, 
and other demographic features played more important roles, %
along with other variables shown in Fig.\ \ref{fig:suppshapfig}.\@

\subsubsubsection{Length of stay}
LOS was predicted in terms of number of days and was binarized at various thresholds. 
LOS in days was predicted within 3.97 days measured by RMSE. 
LOS over 5 days was predicted with an AUC of 0.84 (Fig.\ \ref{fig:roclos5d}), 
and a BSL of 0.15 (calibration curve shown in Fig.\ \ref{fig:callos5d}).\@ 
The most impactful features for the regression and classification problems were similar, 
and included the type of admission, primary diagnosis code, patient age, admission source, 
LOS of the most recent prior admission, medications administered in the hospital in the first 24 hours, 
insurance, and early admission to the intensive care unit, 
among others shown in Figs. \ \ref{fig:shapsumbarlos5d} and~\ref{fig:shapsumlos5d}.\@
Impactful features for length of stay at thresholds of 3 and 7 days are shown in %
Figs.\ \ref{fig:shaplos3d} and~\ref{fig:shaplos7d}, respectively.\@ 
The AUC did not differ in these time points compared to 5 days (\hyperref[table:table2]{Table 2}).\@

\subsubsubsection{Death within 48\textendash72 hours}
Prediction of death within 48--72 hours of admission was predicted with an AUC of 0.81 (\hyperref[table:table2]{Table 2}). 
However, due to extreme class imbalance (e.g.\ in the testing set there were 260,518 non-deaths and 390 deaths), 
this was achieved by simply predicting non-death in every case (see Supplementary Appendix). 
Strategies to produce a reliable model by addressing class imbalance, such as data oversampling, were unsuccessful. 
AUC does not reliably indicate model performance and applicability in this clinical setting.

\subsubsection{Prediction of Patient Characteristics}
To further illustrate the power of explainable ML, 
we predicted key sociodemographic characteristics and explored, for example, 
whether chronic diseases (or patterns of diagnosis) are more prevalent in patients in a given group.
\subsubsubsection{{Gender}}
Gender prediction achieved an AUC of 0.88 (Fig.\ \ref{fig:aucgender}),\@
and a BSL of 0.14 (calibration curve shown in Fig.\ \ref{fig:calgender}).\@
The most impactful features for gender prediction are shown in Fig.\ \ref{fig:shapsumgender}.\@

\subsubsubsection{{Race}}
Race prediction (white vs.\ non-white) achieved an AUC of 0.92 (Fig.\ \ref{fig:aucrace}), 
and a BSL of 0.09 (calibration curve shown in Fig.\ \ref{fig:calrace}).\@
The most impactful features for racial prediction are shown in Fig.\ \ref{fig:shapsumrace}.\@

\subsubsubsection{{Financial Class}}
Payer class prediction (Medicare/Medicaid vs.\ private insurance) achieved an AUC of 0.92 (Fig.\ \ref{fig:aucinsurance}), 
and a BSL of 0.1 (calibration curve shown in Fig.\ \ref{fig:calinsurance}).\@
The most impactful features for insurance prediction are shown in Fig.\ \ref{fig:shapsuminsurance}.\@

\subsubsubsection{{Age}}
When predicted as a continuous variable, age was only able to be predicted within about 10 years (\hyperref[table:table2]{Table 2}).\@
If age was binarized into patients older or younger than 65 years, 
prediction achieved an AUC of 0.94 (Fig.\ \ref{fig:aucage}) and BSL of 0.09 (Fig.\ \ref{fig:calage}).\@
The most impactful features for age prediction are shown in Fig.\ \ref{fig:shapsumage}.\@

\subsubsection{{Variable Interactions}}
SHAP analysis also allows examination of interactions between variables.\@
Key variable interactions are shown in Figs.\ \ref{fig:30dint},~\ref{fig:los5dint},%
~\ref{fig:ageint},~\ref{fig:raceint},~\ref{fig:genderint}, and~\ref{fig:insuranceint}.\@ 
For example, high and low values of heart rate were shown to affect probability of readmission
differently for patients at different ages. A lower heart rate, between 50 and 100 beats per minute,
reduced the likelihood of readmission for older patients, while a high heart rate increased it.\@
For children, a high heart rate had little impact on the model output (Fig.\ \ref{fig:30dinthrage}).\@
For prediction of LOS over 5 days, for middle-aged patients 
the number of medications administered in the first 24 hours had little impact on model output, 
whereas a very low number decreased the prediction for older patients 
and a very high number increased the prediction for younger patients (Fig.\ \ref{fig:los5dintagemeds}).\@
